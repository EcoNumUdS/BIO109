% Please add the following required packages to your document preamble:
% \usepackage{multirow}
% \usepackage[table,xcdraw]{xcolor}
% If you use beamer only pass "xcolor=table" option, i.e. \documentclass[xcolor=table]{beamer}
% \usepackage{longtable}
% Note: It may be necessary to compile the document several times to get a multi-page table to line up properly
\begin{longtable}{lllllll}
  \hline
  \multicolumn{1}{|c|}{} &
    \multicolumn{1}{c|}{} &
    \multicolumn{5}{c|}{Échelle descriptive} \\ \cline{3-7} 
  \multicolumn{1}{|c|}{\multirow{-2}{*}{Critères}} &
    \multicolumn{1}{c|}{\multirow{-2}{*}{Indicateurs}} &
    \multicolumn{1}{c|}{Échec} &
    \multicolumn{1}{c|}{Passable} &
    \multicolumn{1}{c|}{Bien} &
    \multicolumn{1}{c|}{Très bien} &
    \multicolumn{1}{c|}{Excellent} \\ \hline
  \endfirsthead
  %
  \endhead
  %
  \hline
  \endfoot
  %
  \endlastfoot
  %
  Respect des bonnes pratiques de programmation &
    La programmation de la simulation prend la forme d'une fonction ré-utilisable &
    \multicolumn{1}{r}{} &
     &
     &
     &
     \\
   &
    Le programme est divisé en blocs conhérents reflétant la structure conceptuelle des opérations &
     &
     &
     &
     &
     \\
   &
    Respect des pratiques d'indentation &
     &
     &
     &
     &
     \\
   &
    Le nom des scripts permet facilement d'en comprendre la fonction &
     &
     &
     &
     &
     \\
   &
    Les tirages au sort sont contraints et permettent la reproducibilité &
     &
     &
     &
     &
     \\
   &
    Une seule opération par ligne &
     &
     &
     &
     &
     \\
   &
    Le nom donné aux variables permet facilement d'en comprendre la fonction &
     &
     &
     &
     &
     \\
   &
    Les commentaires permettent à un utilisateur externe de comprendre facilement la fonction du bout de code &
    \multicolumn{1}{r}{} &
     &
     &
     &
     \\
   &
    INDEXATION DES JEUX DE DONNÉES + APPLICATION FONCTION À DES GROUPES &
     &
     &
     &
     &
     \\
   &
    Optimisation &
     &
     &
     &
     &
    Vectorisation des opérations \\
   &
    L'organisation du ou des scripts en répertoire permet de facilement retrouver l'information requise &
     &
     &
     &
     &
     \\
  \rowcolor[HTML]{DAE8FC} 
  Résultat.                                                         /40 &
     &
     &
     &
     &
     &
     \\
   &
    Transformation des données sous forme d'états &
    \multicolumn{1}{r}{} &
     &
     &
     &
     \\
   &
    Réalisation de la simulation stochastique (tirage des transitions) &
     &
     &
     &
     &
     \\
  \multirow{-3}{*}{Rigueur de la structure du programme} &
    Réalisation de la figure &
     &
     &
     &
     &
     \\
  \rowcolor[HTML]{DAE8FC} 
  Résultat.                                                         /40 &
     &
     &
     &
     &
     &
     \\
   &
     &
    \multicolumn{1}{r}{} &
     &
     &
     &
     \\
   &
    Le programme produit les résultats attendus &
     &
     &
     &
     &
    \multicolumn{1}{l|}{} \\
  \multirow{-3}{*}{Fonctionnement adéquat du programme} &
    Le programme s'exécute de A à Z sans intervention &
     &
     &
     &
     &
    \multicolumn{1}{l|}{} \\
  \rowcolor[HTML]{DAE8FC} 
  \multicolumn{1}{|l}{\cellcolor[HTML]{DAE8FC}Résultat.                                                         /20} &
     &
     &
     &
     &
     &
    \multicolumn{1}{l|}{\cellcolor[HTML]{DAE8FC}} \\ \hline
  \multicolumn{1}{|l|}{} &
    \multicolumn{6}{l|}{} \\
  \multicolumn{1}{|l|}{\multirow{-2}{*}{Total                                                               /100}} &
    \multicolumn{6}{l|}{} \\ \cline{1-1}
  \multicolumn{1}{|l|}{} &
    \multicolumn{6}{l|}{} \\
  \multicolumn{1}{|l|}{NOTE FINALE                                                /80} &
    \multicolumn{6}{l|}{} \\
  \multicolumn{1}{|l|}{} &
    \multicolumn{6}{l|}{\multirow{-5}{*}{Commentaires}} \\ \hline
  \caption{Excellent : Réalisation remarquable qui surprend par sa qualité et son originalité. Cette production dépasse les exigences et les attentes.
  Très bien : Correspond à un travail impeccable et minutieux, sans erreur.
  Bien : Correspond à un travail moyen, incomplet, où subsistent plusieurs erreurs. Il est nécessaire d’améliorer plusieurs points du travail avant de prétendre à une meilleure appréciation​.
  Passable et échec : Travaux nettement incomplets à plusieurs égards selon les critères d’évaluation. Des erreurs graves peuvent même signifier l’échec.​ }
  \label{tab:my-table}\\
  \end{longtable}