\documentclass[12]{article}
\usepackage[utf8]{inputenc}
\usepackage{array}
\usepackage{caption}
\usepackage{authblk}
\usepackage{geometry}
\usepackage{amsmath}
\usepackage{hyperref}
\usepackage[french]{babel}
\usepackage{url}
\usepackage[french]{isodate}

\geometry{letterpaper,margin=2.5cm}

\hypersetup
{
    colorlinks = true, linkcolor = blue, citecolor = blue, urlcolor = blue,
}

%\def\labelitemi{$\bullet$}


\title{BIO109 : Introduction à la programmation scientifique}
\date {Hiver 2023}
\author {Victor Cameron}

\affil {Département de biologie \\
Université de Sherbrooke \\
Local D8-0012 \\
819-821-8000 \#61928}
\affil {\url{victor.cameron@usherbrooke.ca}}

\begin{document}

	\maketitle

	%-----------------------------
	\section*{Renseignements généraux\footnote{Ce plan de cours est soumis au \textit{Règlement facultaire d'évaluation des apprentissages des étudiantes et des étudiants} de la faculté des sciences de l'Université de Sherbrooke et y est conforme.}} 
        
        \begin{center}
        \begin{tabular}{ m{22em} m{24em} } 
         \hline
         \hline
         \textbf{Nombre de crédits} & 1 \\ 
         \hline
         \textbf{Cours préalables} & Aucun \\
         \hline
         \textbf{Lieu du cours} & D7-2023 \\
         \hline
         \textbf{Jours et heures des cours} & 10, 17, 24 et 31 janvier et 7 février \\ & Tous les cours sont le mardi de 8h30 à 11h30 \\
         \hline
         \textbf{Session} & Hiver 2023 \\
         \hline
         \textbf{Date de début} & 10 janvier \\
         \hline
         \textbf{Date de fin} & 7 février \\
         \hline
         \textbf{Date de remise de l'évaluation finale} & 21 février à 23h59 \\
         \hline
         \textbf{Date limite de retrait} & 16 janvier \\
         \hline
         \textbf{Date limite d'abandon} & 30 janvier \\
         \hline
         \textbf{Assistants à l'enseignement} & Gabriel Bergeron \\ & Local D8-0022 \\ & \url{gabriel.bergeron3@usherbrooke.ca} \\ &
         \\ & Benjamin Mercier \\ & Local D8-0022 \\ & \url{benjamin.b.mercier@usherbrooke.ca} \\
         \hline
         \hline
        \end{tabular}
        \end{center}
	
	\section*{Objectif général}

	Les outils informatiques sont utilisés de façon croissante en écologie,
	que ce soit pour la réalisation d'analyses spatiales, statistiques ou pour
	la gestion de bases de données. La programmation scientifique intervient à
	de nombreuses étapes au cours du baccalauréat en biologie, notamment pour
	l'analyses des données écologiques et pour la réalisation de modèles
	de simulations. Au terme de ce cours, l'étudiant sera en mesure de
	conceptualiser un problème qui requiert de la programmation scientifique
	et de réaliser des tâches courantes de programmation. L'emphase du cours
	sera donné sur les bonnes pratiques de programmation scientifique. Le
	langage de programmation utilisé sera R, mais les techniques de
	programmation de base qui seront enseignées sont transposables à tout
	autre langage utilisé couramment en science.

	%-----------------------------
	\section*{Objectifs spécifiques}

	Au terme de ce cours, l'étudiant sera en mesure de:

	\begin{itemize}
	\renewcommand{\labelitemi}{$\bullet$}

	\item Charger des données et exporter des résultats d'analyses au moyen du
	logiciel R;

	\item Conceptualiser un problème au moyen de pseudo-code;

	\item Manipuler des données;

	\item Rédiger des fonctions;

    \item Programmer des algorithmes afin de réaliser des tâches complexes,
	incluant des boucles et des énoncés conditionnels;

	\item Réaliser des simulations de Monte Carlo;

	\end{itemize}

	%-----------------------------
	\section*{Pré-requis}

    Un ordinateur portable personnel est requis. \\

	Ce cours obligatoire est offert aux étudiants en début de programme de
	baccalauréat en biologie, concentration écologie. Aucun pré-requis n'est
	exigé pour ce cours. Ce cours sera cependant un pré-requis au cours
	BIO-500 sur les méthodes computationnelles en écologie, ainsi qu'au cours
	BIO-300 en biométrie assistée par ordinateur. R est également utilisé
	dans le cours ECL - 510.

	%-----------------------------
	\section*{Approche pédagogique}

	Les connaissances requises pour la programmation scientifique sont
	minimales, l'apprentissage porte davantage sur l'acquisition de
	compétences et le développement de capacités à la résolution de problèmes.
	Les séances seront constituées de courtes leçons magistrales sur des
	notions de bases de programmation, entre-coupées d'exercices spécifiques
	destinés à pratiquer les éléments enseignés. Les séances se conclueront
	sur la réalisation d'un exercice intégrateur à compléter à la maison.

	L'ensemble du matériel du cours sera disponible sur un dépôt git à l'adresse :\\
	\url{https://github.com/EcoNumUdS/BIO109.git}

	%-----------------------------
	\section*{Déroulement du cours}

	\begin{center}
        \begin{tabular}{| p{0.1\linewidth} | p{0.3\linewidth} | p{0.3\linewidth} | p{0.3\linewidth} | } 
        \hline
        Dates & Contenus & Activités pédagogiques & Activités d'évaluation \\ [0.5ex] 
        \hline\hline
        %------------ÉTAPE 0-----------------
        %% Dates
        \textbf{Étape 0} \linebreak 5 au 9 janvier & 
        %% Contenus
        Familiarisation avec l'environnement Moodle\hfill\hfill \linebreak\linebreak 
        	- Lecture des guides d’aide disponibles sur la plateforme\hfill\hfill &
        %% Stratégie pédagogique
        - Exploration de la plateforme Moodle et du dépôt Git\hfill\hfill \linebreak
        - Lecture du plan de cours\hfill\hfill \linebreak
        - Participation au forum sur les attentes par rapport au cours\hfill\hfill & 
        %% Activité d'évaluation
         \\
         \hline
         %------------ÉTAPE 1-----------------
         %% Dates
         \textbf{Étape 1} \linebreak 10 au 16 janvier & 
         %% Contenus
        Introduction à la programmation scientifique\hfill\hfill \linebreak\linebreak
            - Présentation du plan de cours\hfill \linebreak
			- Historique et motivation au calcul scientifique\hfill\hfill \linebreak
        	- Le pseudo-code\hfill\hfill \linebreak
    		- Bonnes pratiques de programmation\hfill\hfill & 
        %% Stratégie pédagogique
        - \textbf{Séance 1} (10 janvier)\hfill\hfill \linebreak
        - Participation au quiz : programme, expérience de programmation, perception d'utilité\hfill\hfill\hfill \linebreak
        - Exercice : computation sur Excel vs. R\hfill\hfill\hfill \linebreak
        - Exercice : pseudo-code\hfill\hfill\hfill \linebreak & 
        %% Activité d'évaluation
        \textbf{Évaluation 1} (individuelle) \linebreak
        \textbf{Conception d'un pseudo-code}\hfill\hfill \linebreak
        Conception d'un pseudo-code permettant de distinguer les étapes d'un programme. Remise sur Moodle avant la scéance 2\hfill\hfill \\
        \hline
        %------------ÉTAPE 2-----------------
    	%% Dates
        \textbf{Étape 2} \linebreak 17 au 23 janvier & 
        %% Contenus
        Les bases du langage R\hfill\hfill \linebreak\linebreak 
            - Interagir avec R\hfill\hfill \linebreak
            - Lire et écrire des fichiers\hfill\hfill \linebreak
            - Le script R\hfill\hfill \linebreak
            - Manipulation des objects\hfill\hfill &
        %% Stratégie pédagogique
        - \textbf{Séance 2} (17 janvier)\hfill\hfill \linebreak
        - Lecture de la rubrique \textit{Les fichiers avec R} sous l'étape 2 de Moodle\hfill\hfill \linebreak
        - Retour sur l'exercice du cours précédent\hfill\hfill \linebreak
        - Exercice : interaction avec la ligne de commande\hfill\hfill \linebreak
        - Exercice : création et indexation de vecteurs\hfill\hfill \linebreak
        - Exercice : création et operation sur matrices\hfill\hfill \linebreak
        - Exercice : manipulation des données avec Excel et R\hfill\hfill &
        %% Activité d'évaluation
        \textbf{Évaluation 2} (individuelle) \linebreak
        \textbf{Conception d'un premier programme}\hfill\hfill \linebreak
        Conception d'un script qui réalisera un ensemble d'étapes de la lacture
        des données à l'enregistrement du tableau final. Remise sur Moodle
        avant la séance 3\hfill\hfill\hfill \\
        \hline
        %------------ÉTAPE 3-----------------
         %% Dates
        \textbf{Étape 3} \linebreak 24 au 30 janvier & 
        %% Contenus
        Opérations et fonctions\hfill\hfill \linebreak\linebreak 
            - Opérations mathématiques\hfill\hfill \linebreak
            - L'anatomie d'une fonction\hfill\hfill \linebreak
            - Automatisation d'une série d'opérations\hfill &
        %% Stratégie pédagogique
        - \textbf{Séance 3} (24 janvier)\hfill\hfill \linebreak 
        - Retour sur l'exercice du cours précédent\hfill\hfill \linebreak
        - Exercice : opérations mathématiques\hfill\hfill & 
        %% Activité d'évaluation
        \textbf{Évaluation 3} (individuelle) \linebreak
        \textbf{Conception d'une fonction}\hfill\hfill \linebreak
    Conception et utilisation d'une fonction automatisant une série d'actions à remettre avant la séance 4\hfill\hfill \\
        \hline
        %------------ÉTAPE 4-----------------
         %% Dates
        \textbf{Étape 4} \linebreak 31 janvier au 6 février & 
        %% Contenus
        Algorithmique I\hfill\hfill \linebreak\linebreak 
            - Boucles\hfill\hfill \linebreak
            - Opérateurs logiques\hfill\hfill &
        %% Stratégie pédagogique
        - \textbf{Séance 4} (31 janvier)\hfill\hfill \linebreak
        - Exercices : Boucles for, boucles while\hfill\hfill \linebreak
        - Exercice : opérateurs logiques\hfill\hfill & 
        %% Activité d'évaluation
        \\
        \hline
        %------------ÉTAPE 5-----------------
         %% Dates
        \textbf{Étape 5} \linebreak 31 janvier au 21 février & 
        %% Contenus
        Algorithmique II\hfill\hfill \linebreak\linebreak 
            - Simulations stochastiques\hfill\hfill \linebreak
            - Optimisation des scripts\hfill\hfill \linebreak\linebreak
        Programmation scientifique en fonction des techniques de bases et des bonnes pratiques &
        %% Stratégie pédagogique
        - Présentation des consignes du travail final (31 janvier)\hfill\hfill \linebreak
        - Consultation de la grille d'évaluation pour l'évaluation terminale\hfill\hfill \linebreak
        - \textbf{Séance 5} (7 février)\hfill\hfill \linebreak
        - Exercice : programme qui pige au hasard\hfill\hfill \linebreak
        - Exercice : échantillonnage d'une distribution\hfill\hfill \linebreak
        - Exercice : Simuler un tirage de Bernouilli\hfill\hfill &
        %% Activité d'évaluation
        \textbf{Évaluation 4} (équipes de 2) \linebreak
        \textbf{Conception d'un programme et d'un pseudo-code permettant d’évaluer la compétence du cours}\hfill\hfill \linebreak\linebreak
        Formatif : À faire après la sécance 4. Remise avant la scéance 5 (7 février, 8h29)\hfill\hfill \linebreak\linebreak
        Sommatif : Remise le 21 février sur Moodle du pseudo-code bonifié et du programme pour l'évaluation terminale \hfill\hfill  \\
         \hline
         \hline
        \end{tabular}
        \end{center}

	%-----------------------------
	\section*{Évaluation}

	L'évaluation porte sur la participation aux exercices (18\%) et sur un
	travail final (82\%). Un exercice simple sera présenté à la fin des
	séances 1-3 et chaque étudiant devra remettre la solution de l'exercice sous
	forme de script avant le début de la séance suivante. Les exercices
	peuvent être réalisés en groupe, mais chaque étudiant devra remettre sa
	propre copie, personnalisée. Les points sont attribués pour la
	participation.

	L'évaluation finale portera sur la réalisation d'un projet de
	programmation en équipe de 2 à remettre deux semaines après la fin du
	dernier cours, soit au plus tard le \textbf{21 février 2022 à 23:59}. La
	pénalité sera de 10\% par jour de retard. Le rapport final sera évalué à
	partir de i) le pseudo-code pour le projet de programmation, ii) le
	respect des bonnes pratiques de programmation et iii) la réussite de
	l'exercice demandé. Les étudiants devront remettre le script nécessaire à
	la réalisation du projet.

    \subsection*{Modalités de remise}

	Les travaux devront tous être remis sur Moodle. Aucun travail ne sera
	accepté par courrier électronique.

	\subsection*{Modalités de correction et de notation pour l'évaluation terminale}

	La note obtenue pour l’ensemble des travaux sera convertie en fonction des
	cotes proposées par la Politique d’évaluation de l’Université de Sherbrooke.
	La notation définitive sera exprimée en conformité avec le règlement de la
	Faculté des sciences de l’Université de Sherbrooke, soit à partir du tableau
	suivant :
	
	\begin{center}
		\begin{table}[h]
        \begin{tabular}{| p{0.2\linewidth} | p{0.2\linewidth} | p{0.2\linewidth} | p{0.2\linewidth} | p{0.2\linewidth} | } 
        \hline
		% Header ------------
        \textbf{Excellent} \linebreak A+, A, A- &
		\textbf{Très bien}\hfill\hfill \linebreak B+, B, B- & 
		\textbf{Bien} \linebreak C+, C, C- & 
		\textbf{Passable} \linebreak D+, D &
		\textbf{Échec} \linebreak E \\ [0.5ex] 
        \hline\hline
		% Body ------------
		% A+ : 91\% et plus \hfill\hfill \linebreak A : 88 à 90\% \hfill\hfill \linebreak A- : 85 à 87\% \hfill\hfill & 
		% B+ : 82 à 84\% \hfill\hfill \linebreak B : 79 à 81\% \hfill\hfill \linebreak B- : 76 à 78\% \hfill\hfill & 
		% C+ : 73 à 75\% \hfill\hfill \linebreak C : 70 à 72\% \hfill\hfill \linebreak C- : 67 à 69\% \hfill\hfill & 
		% D+ : 64 à 66\% \hfill\hfill \linebreak D : 60 à 63\% \hfill\hfill & 
		% E : 0 à 59\% \hfill \\ 
		% \hline
		% \hline
		% Footer ------------
        \multicolumn{2}{l}{W : échec par abandon\hfill\hfill \linebreak
            AB* : abandon\hfill\hfill \linebreak} &
        \multicolumn{2}{l}{
            IN** : Incomplet\hfill\hfill \linebreak 
            R : réussite\hfill\hfill} \\
		% \hline
		\hline
		\end{tabular}
		\caption*{* La mention AB est consignée seulement si l’étudiante ou l’étudiant abandonne le cours avant la date
		limite d'abandon. Si la date limite d’abandon est dépassée, la mention au relevé de notes de l’étudiante
		ou de l’étudiant sera W.\\
		** La metion IN est utilisée au relevé de notes pour les activités
		pédagogiques lorsque, pour des motifs acceptés par la faculté ou le
		centre universitaire de formation, l’étudiante ou l’étudiant n’a pas
		satisfait à toutes les exigences. Est remplacée par la note W (échec
		par abandon) au relevé de notes du trimestre au cours duquel prend fin
		le délai accordé si l’activité n’a pas été complétée.}
		\end{table}
	\end{center}

    La note finale du cours sera remise au plus tard deux semaines après le
	dépôt de l'évaluation finale.

    \subsection*{Appréciation de la qualité de la langue}

    En conformité avec l'article 17 du règlement facultaire d'évaluation
    des apprentissages des étudiantes et des étudiants, la qualité du français
    écrit dans l’évaluation peut être pis en compte. Tout travail non conforme
    aux exigences quant à la qualité du français écrit et aux normes de
    présentation  peut retourné à l'étudiante ou à l'étudiant et peut aussi
    entrainer la perte de points pour une mauvaise qualité du français écrit.
    La qualité du français peut compter jusqu'à 5\% des points alloués à
    l'évaluation.

	
	
    %-----------------------------
	% \section*{Modalités de révision de notes}
    
    

	%-----------------------------
	\section*{Références}

	\begin{itemize}
	\renewcommand{\labelitemi}{$\bullet$}

		\item Paradis, E. 2005. R pour les débutants.
		\\ ftp://cran.r-project.org/pub/R/doc/contrib/Paradis-rdebuts\_fr.pdf

		\item Venables et al. 2016. An Introduction to R.
		\\ https://cran.r-project.org/

		\item Wickham, H and Grolemund, G. 2017. R for Data Science. O'Reilly Media Inc, Sebastopol.

		\item Centre de la Science de la Biodiversité du Québec. Ateliers R du CSBQ. \\ http://qcbs.ca/wiki/r

		\item Crawley, M.J. 2013. The R Book. John Wiley and Sons, Chichester.

		\item Adler, J. 2011. R L'essentiel. Pearson Education France, Paris.

	\end{itemize}

\end{document}
