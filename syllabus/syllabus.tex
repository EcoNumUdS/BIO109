\documentclass[12]{article}
\usepackage[utf8]{inputenc}
\usepackage{authblk}
\usepackage{geometry}
\usepackage{amsmath}
\usepackage{hyperref}
\usepackage[french]{babel}
\usepackage{url}
\usepackage[french]{isodate}

\geometry{letterpaper,margin=2.5cm}

\hypersetup
{
    colorlinks = true, linkcolor = blue, citecolor = blue, urlcolor = blue,
}

%\def\labelitemi{$\bullet$}


\title{BIO-109 \\ Introduction à la programmation scientifique}
\date {\printdate{2018-1-8}}
\author {Dominique Gravel}

\affil {Département de biologie \\
Université de Sherbrooke \\
Local D8-3066 \\
819-821-8000 \#66589}
\affil {\url{dominique.gravel@usherbrooke.ca}}

\begin{document}

	\maketitle

	%-----------------------------
	\section*{Objectif général}

	Les outils informatiques sont utilisés de façon croissante en écologie,
	que ce soit pour la réalisation d'analyses spatiales, statistiques ou pour
	la gestion de bases de données. La programmation scientifique intervient à
	de nombreuses étapes au cours du baccalauréat en biologie, notamment pour
	l'analyses des données écologiques et pour la réalisation de modèles
	de simulations. Au terme de ce cours, l'étudiant sera en mesure de
	conceptualiser un problème qui requiert de la programmation scientifique
	et de réaliser des tâches courantes de programmation. L'emphase du cours
	sera donné sur les bonnes pratiques de programmation scientifique. Le
	langage de programmation utilisé sera R, mais les techniques de
	programmation de base qui seront enseignées sont transposables à tout
	autre langage utilisé couramment en science.

	%-----------------------------
	\section*{Objectifs spécifiques}

	Au terme de ce cours, l'étudiant sera en mesure de:

	\begin{itemize}
	\renewcommand{\labelitemi}{$\bullet$}

	\item Charger des données et exporter des résultats d'analyses au moyen du
	logiciel R;

	\item Conceptualiser un problème au moyen de pseudo-code;

	\item Manipuler des données;

	\item Rédiger des fonctions;

    \item Programmer des algorithmes afin de réaliser des tâches complexes,
	incluant des boucles et des énoncés conditionnels;

	\item Réaliser des simulations de Monte Carlo;

	\end{itemize}

	%-----------------------------
	\section*{Pré-requis}

	Ce cours obligatoire est offert aux étudiants en début de programme de
	baccalauréat en biologie, concentration écologie. Aucun pré-requis n'est
	exigé pour ce cours. Ce cours sera cependant un pré-requis au cours
	BIO-500 sur les méthodes computationnelles en écologie, ainsi qu'au cours
	BIO-300 en biométrie assistée par ordinateur. R est également utilisé
	dans le cours ECL - 510.

	%-----------------------------
	\section*{Approche pédagogique}

	Les connaissances requises pour la programmation scientifique sont
	minimales, l'apprentissage porte davantage sur l'acquisition de
	compétences et le développement de capacités à la résolution de problèmes.
	Les séances seront constituées de courtes leçons magistrales sur des
	notions de bases de programmation, entre-coupées d'exercices spécifiques
	destinés à pratiquer les éléments enseignés. Les séances se conclueront
	sur la réalisation d'un exercice intégrateur à compléter à la maison.

	L'ensemble du matériel du cours sera disponible sur un dépôt git à l'adresse :\\
	\url{https://github.com/EcoNumUdS/BIO109.git}

	%-----------------------------
	\section*{Contenu}

	\subsection*{Cours 1: Introduction à la programmation scientifique}
	\begin{itemize}
	\renewcommand{\labelitemi}{$\bullet$}
		\item Présentation du plan de cours
		\item Historique et motivation au calcul scientifique
		\item Le pseudo-code
		\item Bonnes pratiques de programmation
		\item Installation de R Studio
	\end{itemize}

	\subsection*{Cours 2: Les bases du langage R}
	\renewcommand{\labelitemi}{$\bullet$}
	\begin{itemize}
		\item Interagir avec R
		\item Lire et écrire des fichiers
		\item Le script R
		\item Manipulation des objets
	\end{itemize}

	\subsection*{Cours 3: Opérations et fonctions}
	\renewcommand{\labelitemi}{$\bullet$}
	\begin{itemize}
		\item Opérations mathématiques
		\item L'anatomie d'une fonction
		\item Automatisation d'une série d'opérations
	\end{itemize}

	\subsection*{Cours 4: Algorithmique I}
	\renewcommand{\labelitemi}{$\bullet$}
	\begin{itemize}
		\item Boucles
		\item Opérateurs logiques
	\end{itemize}

	\subsection*{Cours 5: Algorithmique II}
	\renewcommand{\labelitemi}{$\bullet$}
	\begin{itemize}
		\item Simulations de Monte Carlo
		\item Optimisation des scripts
	\end{itemize}

	%-----------------------------
	\section*{Évaluation}

	L'évaluation porte sur la participation aux exercices (20\%) et sur un
	travail final (80\%). Un exercice simple sera présenté à la fin des
	séances 1-4 et chaque étudiant devra remettre la solution de l'exercice sous
	forme de script avant le début de la séance suivante. Les exercices
	peuvent être réalisés en groupe, mais chaque étudiant devra remettre sa
	propre copie, personnalisée. Les points sont attribués pour la
	participation.

	L'évaluation finale portera sur la réalisation d'un projet de
	programmation en équipe de 4 à remettre deux semaines après la fin du
<<<<<<< HEAD
	dernier cours, soit au plus tard le \textbf{20 février 2018 à 16:00}. La
=======
	dernier cours, soit au plus tard le \textcolor{red}{\textbf{20 février 2018 à 16:00}}. La
>>>>>>> b09740ea36bfd29f5d708c96002f06ef773cb40f
	pénalité sera de 10\% par jour de retard. Le rapport final sera évalué à
	partir de i) le pseudo-code pour le projet de programmation, ii) le respect
	des bonnes pratiques de programmation et iii) la réussite de l'exercice
	demandé. Les étudiants devront remettre le script nécessaire à la
	réalisation du projet.

	%-----------------------------
	\section*{Références}

	\begin{itemize}
	\renewcommand{\labelitemi}{$\bullet$}

		\item Paradis, E. 2005. R pour les débutants.
		\\ ftp://cran.r-project.org/pub/R/doc/contrib/Paradis-rdebuts\_fr.pdf

		\item Venables et al. 2016. An Introduction to R.
		\\ https://cran.r-project.org/

		\item Wickham, H and Grolemund, G. 2017. R for Data Science. O'Reilly Media Inc, Sebastopol.

		\item Centre de la Science de la Biodiversité du Québec. Ateliers R du CSBQ. \\ http://qcbs.ca/wiki/r

		\item Crawley, M.J. 2013. The R Book. John Wiley and Sons, Chichester.

		\item Adler, J. 2011. R L'essentiel. Pearson Education France, Paris.

	\end{itemize}

\end{document}
